%% bare_conf.tex
%% V1.4a
%% 2014/09/17
%% by Michael Shell
%% See:
%% http://www.michaelshell.org/
%% for current contact information.
%%
%% This is a skeleton file demonstrating the use of IEEEtran.cls
%% (requires IEEEtran.cls version 1.8a or later) with an IEEE
%% conference paper.
%%
%% Support sites:
%% http://www.michaelshell.org/tex/ieeetran/
%% http://www.ctan.org/tex-archive/macros/latex/contrib/IEEEtran/
%% and
%% http://www.ieee.org/

%%*************************************************************************
%% Legal Notice:
%% This code is offered as-is without any warranty either expressed or
%% implied; without even the implied warranty of MERCHANTABILITY or
%% FITNESS FOR A PARTICULAR PURPOSE! 
%% User assumes all risk.
%% In no event shall IEEE or any contributor to this code be liable for
%% any damages or losses, including, but not limited to, incidental,
%% consequential, or any other damages, resulting from the use or misuse
%% of any information contained here.
%%
%% All comments are the opinions of their respective authors and are not
%% necessarily endorsed by the IEEE.
%%
%% This work is distributed under the LaTeX Project Public License (LPPL)
%% ( http://www.latex-project.org/ ) version 1.3, and may be freely used,
%% distributed and modified. A copy of the LPPL, version 1.3, is included
%% in the base LaTeX documentation of all distributions of LaTeX released
%% 2003/12/01 or later.
%% Retain all contribution notices and credits.
%% ** Modified files should be clearly indicated as such, including  **
%% ** renaming them and changing author support contact information. **
%%
%% File list of work: IEEEtran.cls, IEEEtran_HOWTO.pdf, bare_adv.tex,
%%                    bare_conf.tex, bare_jrnl.tex, bare_conf_compsoc.tex,
%%                    bare_jrnl_compsoc.tex, bare_jrnl_transmag.tex
%%*************************************************************************


% *** Authors should verify (and, if needed, correct) their LaTeX system  ***
% *** with the testflow diagnostic prior to trusting their LaTeX platform ***
% *** with production work. IEEE's font choices and paper sizes can       ***
% *** trigger bugs that do not appear when using other class files.       ***                          ***
% The testflow support page is at:
% http://www.michaelshell.org/tex/testflow/



\documentclass[conference,9pt]{IEEEtran}
% Some Computer Society conferences also require the compsoc mode option,
% but others use the standard conference format.
%
% If IEEEtran.cls has not been installed into the LaTeX system files,
% manually specify the path to it like:
% \documentclass[conference]{../sty/IEEEtran}





% Some very useful LaTeX packages include:
% (uncomment the ones you want to load)


% *** MISC UTILITY PACKAGES ***
%
%\usepackage{ifpdf}
% Heiko Oberdiek's ifpdf.sty is very useful if you need conditional
% compilation based on whether the output is pdf or dvi.
% usage:
% \ifpdf
%   % pdf code
% \else
%   % dvi code
% \fi
% The latest version of ifpdf.sty can be obtained from:
% http://www.ctan.org/tex-archive/macros/latex/contrib/oberdiek/
% Also, note that IEEEtran.cls V1.7 and later provides a builtin
% \ifCLASSINFOpdf conditional that works the same way.
% When switching from latex to pdflatex and vice-versa, the compiler may
% have to be run twice to clear warning/error messages.






% *** CITATION PACKAGES ***
%
%\usepackage{cite}
% cite.sty was written by Donald Arseneau
% V1.6 and later of IEEEtran pre-defines the format of the cite.sty package
% \cite{} output to follow that of IEEE. Loading the cite package will
% result in citation numbers being automatically sorted and properly
% "compressed/ranged". e.g., [1], [9], [2], [7], [5], [6] without using
% cite.sty will become [1], [2], [5]--[7], [9] using cite.sty. cite.sty's
% \cite will automatically add leading space, if needed. Use cite.sty's
% noadjust option (cite.sty V3.8 and later) if you want to turn this off
% such as if a citation ever needs to be enclosed in parenthesis.
% cite.sty is already installed on most LaTeX systems. Be sure and use
% version 5.0 (2009-03-20) and later if using hyperref.sty.
% The latest version can be obtained at:
% http://www.ctan.org/tex-archive/macros/latex/contrib/cite/
% The documentation is contained in the cite.sty file itself.






% *** GRAPHICS RELATED PACKAGES ***
%
\ifCLASSINFOpdf
  % \usepackage[pdftex]{graphicx}
  % declare the path(s) where your graphic files are
  % \graphicspath{{../pdf/}{../jpeg/}}
  % and their extensions so you won't have to specify these with
  % every instance of \includegraphics
  % \DeclareGraphicsExtensions{.pdf,.jpeg,.png}
\else
  % or other class option (dvipsone, dvipdf, if not using dvips). graphicx
  % will default to the driver specified in the system graphics.cfg if no
  % driver is specified.
  % \usepackage[dvips]{graphicx}
  % declare the path(s) where your graphic files are
  % \graphicspath{{../eps/}}
  % and their extensions so you won't have to specify these with
  % every instance of \includegraphics
  % \DeclareGraphicsExtensions{.eps}
\fi
% graphicx was written by David Carlisle and Sebastian Rahtz. It is
% required if you want graphics, photos, etc. graphicx.sty is already
% installed on most LaTeX systems. The latest version and documentation
% can be obtained at: 
% http://www.ctan.org/tex-archive/macros/latex/required/graphics/
% Another good source of documentation is "Using Imported Graphics in
% LaTeX2e" by Keith Reckdahl which can be found at:
% http://www.ctan.org/tex-archive/info/epslatex/
%
% latex, and pdflatex in dvi mode, support graphics in encapsulated
% postscript (.eps) format. pdflatex in pdf mode supports graphics
% in .pdf, .jpeg, .png and .mps (metapost) formats. Users should ensure
% that all non-photo figures use a vector format (.eps, .pdf, .mps) and
% not a bitmapped formats (.jpeg, .png). IEEE frowns on bitmapped formats
% which can result in "jaggedy"/blurry rendering of lines and letters as
% well as large increases in file sizes.
%
% You can find documentation about the pdfTeX application at:
% http://www.tug.org/applications/pdftex





% *** MATH PACKAGES ***
%
%\usepackage[cmex10]{amsmath}
% A popular package from the American Mathematical Society that provides
% many useful and powerful commands for dealing with mathematics. If using
% it, be sure to load this package with the cmex10 option to ensure that
% only type 1 fonts will utilized at all point sizes. Without this option,
% it is possible that some math symbols, particularly those within
% footnotes, will be rendered in bitmap form which will result in a
% document that can not be IEEE Xplore compliant!
%
% Also, note that the amsmath package sets \interdisplaylinepenalty to 10000
% thus preventing page breaks from occurring within multiline equations. Use:
%\interdisplaylinepenalty=2500
% after loading amsmath to restore such page breaks as IEEEtran.cls normally
% does. amsmath.sty is already installed on most LaTeX systems. The latest
% version and documentation can be obtained at:
% http://www.ctan.org/tex-archive/macros/latex/required/amslatex/math/





% *** SPECIALIZED LIST PACKAGES ***
%
%\usepackage{algorithmic}
% algorithmic.sty was written by Peter Williams and Rogerio Brito.
% This package provides an algorithmic environment fo describing algorithms.
% You can use the algorithmic environment in-text or within a figure
% environment to provide for a floating algorithm. Do NOT use the algorithm
% floating environment provided by algorithm.sty (by the same authors) or
% algorithm2e.sty (by Christophe Fiorio) as IEEE does not use dedicated
% algorithm float types and packages that provide these will not provide
% correct IEEE style captions. The latest version and documentation of
% algorithmic.sty can be obtained at:
% http://www.ctan.org/tex-archive/macros/latex/contrib/algorithms/
% There is also a support site at:
% http://algorithms.berlios.de/index.html
% Also of interest may be the (relatively newer and more customizable)
% algorithmicx.sty package by Szasz Janos:
% http://www.ctan.org/tex-archive/macros/latex/contrib/algorithmicx/




% *** ALIGNMENT PACKAGES ***
%
%\usepackage{array}
% Frank Mittelbach's and David Carlisle's array.sty patches and improves
% the standard LaTeX2e array and tabular environments to provide better
% appearance and additional user controls. As the default LaTeX2e table
% generation code is lacking to the point of almost being broken with
% respect to the quality of the end results, all users are strongly
% advised to use an enhanced (at the very least that provided by array.sty)
% set of table tools. array.sty is already installed on most systems. The
% latest version and documentation can be obtained at:
% http://www.ctan.org/tex-archive/macros/latex/required/tools/


% IEEEtran contains the IEEEeqnarray family of commands that can be used to
% generate multiline equations as well as matrices, tables, etc., of high
% quality.




% *** SUBFIGURE PACKAGES ***
%\ifCLASSOPTIONcompsoc
%  \usepackage[caption=false,font=normalsize,labelfont=sf,textfont=sf]{subfig}
%\else
%  \usepackage[caption=false,font=footnotesize]{subfig}
%\fi
% subfig.sty, written by Steven Douglas Cochran, is the modern replacement
% for subfigure.sty, the latter of which is no longer maintained and is
% incompatible with some LaTeX packages including fixltx2e. However,
% subfig.sty requires and automatically loads Axel Sommerfeldt's caption.sty
% which will override IEEEtran.cls' handling of captions and this will result
% in non-IEEE style figure/table captions. To prevent this problem, be sure
% and invoke subfig.sty's "caption=false" package option (available since
% subfig.sty version 1.3, 2005/06/28) as this is will preserve IEEEtran.cls
% handling of captions.
% Note that the Computer Society format requires a larger sans serif font
% than the serif footnote size font used in traditional IEEE formatting
% and thus the need to invoke different subfig.sty package options depending
% on whether compsoc mode has been enabled.
%
% The latest version and documentation of subfig.sty can be obtained at:
% http://www.ctan.org/tex-archive/macros/latex/contrib/subfig/




% *** FLOAT PACKAGES ***
%
%\usepackage{fixltx2e}
% fixltx2e, the successor to the earlier fix2col.sty, was written by
% Frank Mittelbach and David Carlisle. This package corrects a few problems
% in the LaTeX2e kernel, the most notable of which is that in current
% LaTeX2e releases, the ordering of single and double column floats is not
% guaranteed to be preserved. Thus, an unpatched LaTeX2e can allow a
% single column figure to be placed prior to an earlier double column
% figure. The latest version and documentation can be found at:
% http://www.ctan.org/tex-archive/macros/latex/base/


%\usepackage{stfloats}
% stfloats.sty was written by Sigitas Tolusis. This package gives LaTeX2e
% the ability to do double column floats at the bottom of the page as well
% as the top. (e.g., "\begin{figure*}[!b]" is not normally possible in
% LaTeX2e). It also provides a command:
%\fnbelowfloat
% to enable the placement of footnotes below bottom floats (the standard
% LaTeX2e kernel puts them above bottom floats). This is an invasive package
% which rewrites many portions of the LaTeX2e float routines. It may not work
% with other packages that modify the LaTeX2e float routines. The latest
% version and documentation can be obtained at:
% http://www.ctan.org/tex-archive/macros/latex/contrib/sttools/
% Do not use the stfloats baselinefloat ability as IEEE does not allow
% \baselineskip to stretch. Authors submitting work to the IEEE should note
% that IEEE rarely uses double column equations and that authors should try
% to avoid such use. Do not be tempted to use the cuted.sty or midfloat.sty
% packages (also by Sigitas Tolusis) as IEEE does not format its papers in
% such ways.
% Do not attempt to use stfloats with fixltx2e as they are incompatible.
% Instead, use Morten Hogholm'a dblfloatfix which combines the features
% of both fixltx2e and stfloats:
%
% \usepackage{dblfloatfix}
% The latest version can be found at:
% http://www.ctan.org/tex-archive/macros/latex/contrib/dblfloatfix/




% *** PDF, URL AND HYPERLINK PACKAGES ***
%
%\usepackage{url}
% url.sty was written by Donald Arseneau. It provides better support for
% handling and breaking URLs. url.sty is already installed on most LaTeX
% systems. The latest version and documentation can be obtained at:
% http://www.ctan.org/tex-archive/macros/latex/contrib/url/
% Basically, \url{my_url_here}.




% *** Do not adjust lengths that control margins, column widths, etc. ***
% *** Do not use packages that alter fonts (such as pslatex).         ***
% There should be no need to do such things with IEEEtran.cls V1.6 and later.
% (Unless specifically asked to do so by the journal or conference you plan
% to submit to, of course. )


% correct bad hyphenation here
\hyphenation{op-tical net-works semi-conduc-tor}

\usepackage{url}
\usepackage{listings}

\begin{document}
%
% paper title
% Titles are generally capitalized except for words such as a, an, and, as,
% at, but, by, for, in, nor, of, on, or, the, to and up, which are usually
% not capitalized unless they are the first or last word of the title.
% Linebreaks \\ can be used within to get better formatting as desired.
% Do not put math or special symbols in the title.
\title{Cloud (INGI2145) : Scaling a Microblogging Cloud Application}


% author names and affiliations
% use a multiple column layout for up to three different
% affiliations
\author{\IEEEauthorblockN{Thibault Gerondal}
%\IEEEauthorblockA{School of Electrical and\\Computer Engineering\\
%Georgia Institute of Technology\\
%Atlanta, Georgia 30332--0250\\
%Email: http://www.michaelshell.org/contact.html}
\and
\IEEEauthorblockN{Nicolas Ooghe}
%\IEEEauthorblockA{Twentieth Century Fox\\
%Springfield, USA\\
%Email: homer@thesimpsons.com}
\and
\IEEEauthorblockN{Lejoly Florent}
\and
\IEEEauthorblockN{Vanden Bulcke C\'edric}
}

%\IEEEauthorblockA{Starfleet Academy\\
%San Francisco, California 96678--2391\\
%Telephone: (800) 555--1212\\
%Fax: (888) 555--1212}}

% conference papers do not typically use \thanks and this command
% is locked out in conference mode. If really needed, such as for
% the acknowledgment of grants, issue a \IEEEoverridecommandlockouts
% after \documentclass

% for over three affiliations, or if they all won't fit within the width
% of the page, use this alternative format:
% 
%\author{\IEEEauthorblockN{Michael Shell\IEEEauthorrefmark{1},
%Homer Simpson\IEEEauthorrefmark{2},
%James Kirk\IEEEauthorrefmark{3}, 
%Montgomery Scott\IEEEauthorrefmark{3} and
%Eldon Tyrell\IEEEauthorrefmark{4}}
%\IEEEauthorblockA{\IEEEauthorrefmark{1}School of Electrical and Computer Engineering\\
%Georgia Institute of Technology,
%Atlanta, Georgia 30332--0250\\ Email: see http://www.michaelshell.org/contact.html}
%\IEEEauthorblockA{\IEEEauthorrefmark{2}Twentieth Century Fox, Springfield, USA\\
%Email: homer@thesimpsons.com}
%\IEEEauthorblockA{\IEEEauthorrefmark{3}Starfleet Academy, San Francisco, California 96678-2391\\
%Telephone: (800) 555--1212, Fax: (888) 555--1212}
%\IEEEauthorblockA{\IEEEauthorrefmark{4}Tyrell Inc., 123 Replicant Street, Los Angeles, California 90210--4321}}




% use for special paper notices
%\IEEEspecialpapernotice{(Invited Paper)}



% make the title area
\maketitle

% As a general rule, do not put math, special symbols or citations
% in the abstract
%\begin{abstract}
%The abstract goes here.
%\end{abstract}

% no keywords




% For peer review papers, you can put extra information on the cover
% page as needed:
% \ifCLASSOPTIONpeerreview
% \begin{center} \bfseries EDICS Category: 3-BBND \end{center}
% \fi
%
% For peerreview papers, this IEEEtran command inserts a page break and
% creates the second title. It will be ignored for other modes.
\IEEEpeerreviewmaketitle



\section{Introduction}
In that project we were asking to scale a micro-blogging Cloud application called \texttt{Ribbit}. On this report, we will explain how we designed a solution for the three parts of the project and the problems we encountered during the development of these solutions (with the solutions of these problems of course), and the technologies we have used to help us develop our solutions. We will first talk about the problem of scaling the frontend, then the scaling of the timeline, to finish with the analytics part of that project.

%\section{Design of our Solution}
\section{Scaling the Frontend}
\subsection{The NodeJS}
In order to scale the application, we first started by adding more NodeJS instances. Now since NodeJS works only on a single core we required the help of \texttt{node-cluster} which allows us to launch multiple NodeJS workers as nodeJS is single threaded. With node-cluster, we can launch as much thread as we have core on the computer. We have created a configuration file where you can specify the number of thread wanted or the value "-1" if you want to use as much threads as the computer have cores.
\subsection{Sticky-Session}
% redis 
A problem occurred when a connected client is balanced between one NodeJS worker to another because the session variables are stored inside the worker itself. To solve that problem, we used Redis, which will keep in memory the session variables instead of the workers, so that a user won't be disconnected when being balanced between the NodeJS instances.

\subsection{NGINX}
Since there are multiple node workers for the application, we use NGINX in reverse proxy mode to dispatch the load toward the different workers.  This can be configured by adding multiple servers line to the nginx configuration.

\begin{lstlisting}
upstream ribbit {
        server localhost:3002;
        # more server directives here..
}
\end{lstlisting}

\subsection{NGINX Bottleneck}
Because all user's requests pass through NGINX first, it can become a bottleneck for our application. In the interest of avoiding this situation, a solution could be to set multiple NGINX servers and configure a DNS round-robin to address them. 
% Ingix with multiple instances of node js 
% redis session user support 
% cluster 
\section{Timeline scaling}
% kafka -> cassandra 
% avoid faneout of write + kafka fault tollerant
%
\subsection{Followers-Following}
In order to represent the relation between users, we have to create two relations: user "X" follows "Y" and user "Y" is followed by "X". To do that, we have created two separate tables named respectively "Forward Following"  and "Backward Following". Since Cassandra is based on a key for retrieving the information, it seems necessary to do this separation.
\subsection{The Tweets}
For the representation of tweets inside Cassandra, we choose to store the tweets according to a \texttt{timeuuid} data type as the primary key. Thanks to this choice we are assured that the retrieve is clustered and that the charge will be load-balanced between the nodes. As we know, Cassandra is a NoSQL distributed database management system, there is no possibility of relationships between tables. From this fact and because we want our system the more scalable possible, we have to design our database in a way that most of the computation is done \textit{a priori}. In our case, we decided to create tables called "Userline" and "Timeline" who respectively represent the list of tweetid of the posted tweets of a user and the newsfeed of a user (tweets posted by users we are following). For the table "Timeline", the computation of who sees which tweets is done when a user posts a tweet to his followers.
The primary key is the tuple (username, timeuuid) which means that the partition key (responsible for data distribution accross nodes) is username and the clustering key (responsible for data sorting within the partition) is the timeuuid. In order to store the Timeline in a smart fashion, we store it by ordering them with the timeuuid column in a decreasing way so that the most recent tweets are stored on the top of the partition. The same idea is in application for the Userline. For querying the Timeline of a user, we first retrieve a list of tweetid from the Timeline table. This list of tweets will be ordered. But when we query the Tweets table to retrieve the full information about them, since it comes from multiple partitions, they will lose the ordering. Nevertheless, we reorder them in NodeJS. Since only 10 tweets at a time are requested thanks to an infinite scrolling way, this operation is cost efficient.

\subsection{Kafka}
Concerning the posting of tweets, since it is the main feature of the application we needed something robust. In order to avoid the fan-in on write, we use Kafka as an intermediate way to pipeline the data to be written on Cassandra. To do so, we implemented a kafka driver in the NodeJS that sends the tweets in a kafka topic (queue). We developped an other piece of software (implemented in Java this time) that implements a cassandra driver and who will be responsible to read the Kafka topic and write these tweets into Cassandra.
We chose to use Kafka as pipeline for the tweet inserts is because Kafka has the properties to be fault-tolerant which, for tweets based application, is an important feature. 
%
% added feature
Nevertheless, we made a distinction between the insert of a tweet inside the author's timeline and its followers timeline. Indeed, because we wanted to have an interactive application adn because statistically a user doesn't post lots of tweets within a short period of time, we decided to insert the tweet in the author's userline within the nodejs. This choice gives a fast feedback to the author of the tweet in the sense that he will see the insert in its userline almost instantaneously. But because a user can potentially have a lot of followers, the insert on the followers timeline is done using Kafka and the java driver for all the reasons stated above. 

\subsection{loader.js : poorly designed}
We modified the loader.js logic to make it synchronous. This was necessary in order not to have any concurrency problems between the functions \texttt{insertUsers} and \texttt{insertTweet}. Indeed, without modification, there was a concurrency problem as we try to write new follower-following relations in the same time that we try to retrieve them. Moreover, we have ensured that the program ends once all requests are executed.

\section{Analytics (Top 10 hashtags)}
To compute the top 10 hashtags tweeted (and retweeted) on the last 10 minutes, we used Storm. Storm comes with two different types of node : the Spout and the Bolt. The Spouts are basically the nodes that will receive the data and emit them as a tuple destined to the Bolt nodes. The Bolt are the ones that will perform the actual computation. In order to pass the data from the NodeJS to the Spout of Storm, we also use a Kafka topic and directly read on the topic within the Spout thanks to a java library. The data emitted by the Spout will be first be processed by a "split" Bolt that will be responsible for extracting all the words containing an hashtag. Then "counter" Bolt will count (using a "rolling count") the number of occurrences on the incoming hashtags. Then as third layer,  "intermediateRanker" Bolts will compute an intermediate ranking on the incoming counter (Note that this intermediateRanker plays a similar role as combiner in Hadoop). Then finally, "totalRanker" Bolt will perform the final ranking on the hashtag. Note that the number of workers for the Bolts is specified inside the configuration file but there should be only one Spout and one "totalRanker" Bolt for this job. The final result of the "totalRanker" is stored inside Redis thanks to another Bolt.
The application can then access the result of the computation directly on the Redis. The result will be updated every minutes. In order to modify the parameters of the computation (such as the top 20 hashtags instead of 10) a configuration file can be found in the resources directory. 
%  kafka -> storm 
%
%

\section{Vagrant \& Puppet}
As there is a lot of things to configure and launch in the supplied Virtual Machine, we created a file with all needed commands to run and also a puppet script that execute them. Commands can be found in the file \texttt{configuration/VM.md}. The puppet file can be found in \texttt{VM/manifests/base.pp}. It is possible to create a Virtual Machine that will execute the puppet manifest by launching the command \texttt{vagrant up --provision} in the directory \texttt{VM}. This puppet file can take a lot of time to execute ($\pm{}$40 min). There are deamons that launch everything needed for this project to work at each boot. As the memory of the Virtual Machine was a problem, we decided to not install any desktop environment and put 4Go of memory on it. To access to the website, we created a port forwarding (host:8080 $\rightarrow{}$ guest:80) between vagrant and the host. Ribbit is reachable via the link \url{http://localhost:8080}. The deamon and configuration files created are the following :
\begin{itemize}
\item /etc/init.d/zookeeper : launches zookeeper with the configuration file : /etc/zoo.cfg. Logs available in /var/log/zoo.log
\item /etc/init.d/kafka launches : kafka with the configuration file : /etc/kafka.properties. Logs available in /var/log/kafka.log
\item /etc/init.d/twitter\_analytics : launches the analytics (trendings) with configuration file : /home/vagrant/cloud\_2/analytics/analyticsConsumer/src/main/resources/application.conf. Logs available in /var/log/analyticsConsumer.log
\item /etc/init.d/twitter\_cassandra : launches the cassandra consumer (configuration in the daemon itself via the arguments). Logs available in /var/log/cassandraConsumer.log
\item /etc/init.d/ribbit : launches the web app ribbit with the configuration file : /home/vagrant/cloud\_2/app/config/production.json. Logs available in /var/log/ribbit.log
\end{itemize}
You might need to restart the Virtual Machine once the provision is done. Every services will be automatically launched at the startup.

\section{AWS EC2}
If you don't want to wait for the Virtual Machine to be provisionned, our project is also deployed on AWS EC2 on the following URL : \url{http://ec2-54-201-0-17.us-west-2.compute.amazonaws.com}. We deployed our project on four small instances. Here is the services running on each server :
\begin{itemize}
\item 54.201.37.158 : Cassandra, Redis, 2 nodeJS
\item 54.201.79.233 : Analytics consumer, Tweets consumer, 2 nodeJS
\item 54.201.19.47 : Kafka, Zookeeper, 2 nodeJS
\item 54.201.0.17 : Nginx, 2 nodeJS
\end{itemize}
For being able to see that requests sent to the nginx are dispatched between these 4 servers, you can check the \texttt{HTTP header X-Upstream} to know the upstream source of any request.

\section{Group work \& planing}
In order to tackle this project, we obviously had to divide the workload within the group members. To do so, since we came up rapidely with missing part in the NodeJS and the solution of the first question (Scaling the frontend), we decided to put two people on the search of documentation and the implementation of the driver that reads from kafka and write to cassandra. And the two other on the search of documentation and implementation of the storm diver which performs the computation of the top $10$ hashtags. Once all that was done and working good with each other on a local machine, we put two people on the report and the two other on the deploying of the solution on Amazon.
 
\section{Conclusion}
This project was very instructive, our group learnt a lot about scaling an existing application in the purpose to deal with Big Data. Also, we discovered and played with a lot of cloud-related applications (Zookeeper, Kafka, Cassandra, Storm, etc.). And this project has taught us more about the Linux ecosystem for being able to put all these services together.

% no \IEEEPARstart

% You must have at least 2 lines in the paragraph with the drop letter
% (should never be an issue)



% An example of a floating figure using the graphicx package.
% Note that \label must occur AFTER (or within) \caption.
% For figures, \caption should occur after the \includegraphics.
% Note that IEEEtran v1.7 and later has special internal code that
% is designed to preserve the operation of \label within \caption
% even when the captionsoff option is in effect. However, because
% of issues like this, it may be the safest practice to put all your
% \label just after \caption rather than within \caption{}.
%
% Reminder: the "draftcls" or "draftclsnofoot", not "draft", class
% option should be used if it is desired that the figures are to be
% displayed while in draft mode.
%
%\begin{figure}[!t]
%\centering
%\includegraphics[width=2.5in]{myfigure}
% where an .eps filename suffix will be assumed under latex, 
% and a .pdf suffix will be assumed for pdflatex; or what has been declared
% via \DeclareGraphicsExtensions.
%\caption{Simulation results for the network.}
%\label{fig_sim}
%\end{figure}

% Note that IEEE typically puts floats only at the top, even when this
% results in a large percentage of a column being occupied by floats.


% An example of a double column floating figure using two subfigures.
% (The subfig.sty package must be loaded for this to work.)
% The subfigure \label commands are set within each subfloat command,
% and the \label for the overall figure must come after \caption.
% \hfil is used as a separator to get equal spacing.
% Watch out that the combined width of all the subfigures on a 
% line do not exceed the text width or a line break will occur.
%
%\begin{figure*}[!t]
%\centering
%\subfloat[Case I]{\includegraphics[width=2.5in]{box}%
%\label{fig_first_case}}
%\hfil
%\subfloat[Case II]{\includegraphics[width=2.5in]{box}%
%\label{fig_second_case}}
%\caption{Simulation results for the network.}
%\label{fig_sim}
%\end{figure*}
%
% Note that often IEEE papers with subfigures do not employ subfigure
% captions (using the optional argument to \subfloat[]), but instead will
% reference/describe all of them (a), (b), etc., within the main caption.
% Be aware that for subfig.sty to generate the (a), (b), etc., subfigure
% labels, the optional argument to \subfloat must be present. If a
% subcaption is not desired, just leave its contents blank,
% e.g., \subfloat[].


% An example of a floating table. Note that, for IEEE style tables, the
% \caption command should come BEFORE the table and, given that table
% captions serve much like titles, are usually capitalized except for words
% such as a, an, and, as, at, but, by, for, in, nor, of, on, or, the, to
% and up, which are usually not capitalized unless they are the first or
% last word of the caption. Table text will default to \footnotesize as
% IEEE normally uses this smaller font for tables.
% The \label must come after \caption as always.
%
%\begin{table}[!t]
%% increase table row spacing, adjust to taste
%\renewcommand{\arraystretch}{1.3}
% if using array.sty, it might be a good idea to tweak the value of
% \extrarowheight as needed to properly center the text within the cells
%\caption{An Example of a Table}
%\label{table_example}
%\centering
%% Some packages, such as MDW tools, offer better commands for making tables
%% than the plain LaTeX2e tabular which is used here.
%\begin{tabular}{|c||c|}
%\hline
%One & Two\\
%\hline
%Three & Four\\
%\hline
%\end{tabular}
%\end{table}


% Note that the IEEE does not put floats in the very first column
% - or typically anywhere on the first page for that matter. Also,
% in-text middle ("here") positioning is typically not used, but it
% is allowed and encouraged for Computer Society conferences (but
% not Computer Society journals). Most IEEE journals/conferences use
% top floats exclusively. 
% Note that, LaTeX2e, unlike IEEE journals/conferences, places
% footnotes above bottom floats. This can be corrected via the
% \fnbelowfloat command of the stfloats package.




%\section{Conclusion}
%The conclusion goes here.




% conference papers do not normally have an appendix


% use section* for acknowledgment
%\section*{Acknowledgment}


%The authors would like to thank...





% trigger a \newpage just before the given reference
% number - used to balance the columns on the last page
% adjust value as needed - may need to be readjusted if
% the document is modified later
%\IEEEtriggeratref{8}
% The "triggered" command can be changed if desired:
%\IEEEtriggercmd{\enlargethispage{-5in}}

% references section

% can use a bibliography generated by BibTeX as a .bbl file
% BibTeX documentation can be easily obtained at:
% http://www.ctan.org/tex-archive/biblio/bibtex/contrib/doc/
% The IEEEtran BibTeX style support page is at:
% http://www.michaelshell.org/tex/ieeetran/bibtex/
%\bibliographystyle{IEEEtran}
% argument is your BibTeX string definitions and bibliography database(s)
%\bibliography{IEEEabrv,../bib/paper}
%
% <OR> manually copy in the resultant .bbl file
% set second argument of \begin to the number of references
% (used to reserve space for the reference number labels box)
%\begin{thebibliography}{1}

%\bibitem{IEEEhowto:kopka}
%H.~Kopka and P.~W. Daly, \emph{A Guide to \LaTeX}, 3rd~ed.\hskip 1em plus
%  0.5em minus 0.4em\relax Harlow, England: Addison-Wesley, 1999.

%\end{thebibliography}




% that's all folks
\end{document}